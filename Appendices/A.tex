\chap{SIA Implementation code}

\section{SIA: Imported libraries}
\label{SIA_libraries}
The library "os" is really important since provides a waay of using operating system dependent functinality.
\begin{lstlisting}
import os
\end{lstlisting}

Also the library "sys" would be very useful for test and execute the program, mainly because it allows to input directly from terminal.
\begin{lstlisting}
import sys
\end{lstlisting}

The "pylab" library will be useful for plot data.
\begin{lstlisting}
import pylab
\end{lstlisting}

The "pandas" library will be very useful for read the data from CSV dataset and setup the plot abut it.
\begin{lstlisting}
import pandas as pd
\end{lstlisting}

The "numpy" library it's used for mathematic purpose, such as calculating the correlation coefficent between two series.
\begin{lstlisting}
import numpy as np
\end{lstlisting}
 
The "pyplot" library it's used for basic graphic displaying and customization, easy to use but very efficent.
\begin{lstlisting}
import matplotlib.pyplot as pyplot
\end{lstlisting}

The library "PIL" supports many file formats, and provides powerful image processing and graphics capabilities.
\begin{lstlisting}
from PIL import Image
\end{lstlisting}

The library "fpdf" allows to generate and use PDF file.
\begin{lstlisting}
from fpdf import FPDF
\end{lstlisting}

\section{SIA section I: Total graphic for all the years}
\label{SIA_section_I}
\textbf{Code implementation:}\\
During this section of the code was used "pandas" library for read the dataset.
\begin{lstlisting}
series = pd.read_csv("Dataset.csv", usecols=[1,sys.argv[1]])
\end{lstlisting}

Then using the "pyplot" library has been possible to setup the plot of the input data.
\begin{lstlisting}
series.plot(color="blue", linewidth=1.5)
\end{lstlisting}


Thera are some settings about the axis x just to display the data in the right format, are easy to change and to costume.
\begin{lstlisting}
years = ["2005","2006","2007","2008","2009","2010",
	"2011","2012","2013","2014","2015","2016"]
x = range(144)
pyplot.xticks(np.arange(min(x), max(x)+1, 12.0), years)
pyplot.title("Total graphic from 2005 to 2016")
\end{lstlisting}

Once setted up the plot of the current data, the next step was to display the trendline of the current graphic. \\ 
The following code represent the method for calculate and display it.
\begin{lstlisting}
def trendline(x, y, col):
	z = np.polyfit(x, y, 1)
	p = np.poly1d(z)
	pylab.plot(x,p(x), c=col)
	# print "y=%.6fx+(%.6f)"%(z[0],z[1])
\end{lstlisting}	

At this point the current data values have been read again and passed to the method just impleneted above for calculating the trendline.
\begin{lstlisting}
series2 = pd.read_csv("Dataset.csv", usecols=[sys.argv[1]],
			squeeze=True)
trendline(x, seriesV.values, "red")
\end{lstlisting}

There is the possibility to save the graphic like an image and/or display it.
\begin{lstlisting}
saveFigure("_Total.jpg")
\end{lstlisting}


\section{SIA section II: Single graphics for each year}
\label{SIA_section_II}
\textbf{Code implementation:}\\
During this section of the code was used "pandas" library for read the dataset.
\begin{lstlisting}
series2 = pd.read_csv("Dataset.csv", index_col=['Month'],
			usecols=[0,1,sys.argv[1]])
\end{lstlisting}

Some initialization of variables that are going to be useful.
\begin{lstlisting}
months = ["Jan","Feb","Mar","Apr","May","Jun",
		"Jul","Aug","Sep","Oct","Nov","Dec"]
x_pos = np.arange(len(months))
test = []
j = 0
\end{lstlisting}

The following code allows the system to split the values and display them in the right way: that means that are going to be splitted for each single year and then plotted on the same graphic.
\begin{lstlisting}
for i in range(len(series2.values)):
	if j in range(12):
		test.append(series2.values[i][1])
		j = j + 1
	else:
		pyplot.plot(x_pos, test, linewidth=2, 
			alpha=0.8, 
			label = int(series2.values[i-1][0]))
		test = []
		test.append(series2.values[i][1])
		j = 1

\end{lstlisting}

These are some personalization settings that could be easily changed as you want.
\begin{lstlisting}
ax.legend(loc=4, ncol=1, fancybox=True, shadow=True)
pyplot.xticks(x_pos,months)
pyplot.xlim(0,11)
pyplot.title(sys.argv[1]+ 
		": Single year's graphic from 2005 to 2016")
\end{lstlisting}

There is the possibility to save the graphic like an image and/or display it.
\begin{lstlisting}
saveFigure("_Years.jpg")
\end{lstlisting}


\section{SIA section III: Correlation matrix between years}
\label{SIA_section_III}
\textbf{Code implementation:}\\
During this section of the code was used "pandas" library for read the dataset.
\begin{lstlisting}
series3 = pd.read_csv("Dataset.csv", index_col=['Year'],
			usecols=[0,sys.argv[1]])
\end{lstlisting}

\begin{lstlisting}
corr = []
test = []
j = 0
# Collecting the correct values to elaborate.
for i in range(len(series2.values)+1):
	if j in range(12):
		test.append(series2.values[i][1])
		j = j + 1
	else:
		corr.append(test)
		test = []
		if i in range(144):
			test.append(series2.values[i][1])
			j = 1
\end{lstlisting}

With the library "numpy" is possible to calculate the correlation coefficents between all the variables in the series just read.
\begin{lstlisting}
test = np.corrcoef(corr)
\end{lstlisting}

Setup the figure that will display the correlation matrix using the library "pypot".
\begin{lstlisting}
fig2 = pyplot.figure()
ax = fig2.add_subplot(111)
\end{lstlisting}

Creating the correlation matrix using the already calculated correlation coefficents.
\begin{lstlisting}
cax = ax.matshow(test, interpolation='nearest')
\end{lstlisting}

Settings for display the matrix in the right way, in particular for the values to display on both the axis x and y, in this case every single year from 2005 to 2016
\begin{lstlisting}
pyplot.title(sys.argv[1]+ ": Correlation between different years")
years = ["2005","2006","2007","2008","2009","2010",
	"2011","2012","2013","2014","2015","2016"]
x_pos = np.arange(len(years))
y_pos = np.arange(len(years))
pyplot.yticks(y_pos,years)
pyplot.xticks(x_pos,years)
pyplot.colorbar(cax)
\end{lstlisting}
\newpage
Adding a title to the graphic that we are going to display and also a bar that works like a legend for the colors of the matrix, allowing the reader to better understand the values reported inside the matrix.
\begin{lstlisting}
saveFigure("_Years_Matrix.jpg")
\end{lstlisting}

There is the possibility to save the correlation matrix like an image and/or display it.
\begin{lstlisting}
pyplot.savefig("OUTPUT_DIRECTORY", format="jpg")
pyplot.show()
\end{lstlisting}


\section{SIA section IV: Correlation matrix between months}
\label{SIA_section_IV}
\textbf{Code implementation:}\\
During this section of the code was used "pandas" library for read the dataset.
\begin{lstlisting}
series4 = pd.read_csv("Dataset.csv", usecols=[0,1,sys.argv[1]])
\end{lstlisting}

\begin{lstlisting}
corr = []
for Month, Year in series4.groupby(["Month"], sort=False):
	corr.append(Year[sys.argv[1]].values)
\end{lstlisting}
	
With the library "numpy" is possible to calculate the correlation coefficents between all the variables in the series just read.
\begin{lstlisting}
test = np.corrcoef(series4.values)
\end{lstlisting}

Setup the figure that will display the correlation matrix using the library "pypot".
\begin{lstlisting}
fig2 = pyplot.figure()
ax = fig2.add_subplot(111)
\end{lstlisting}

Creating the correlation matrix using the already calculated correlation coefficents.
\begin{lstlisting}
cax = ax.matshow(test, interpolation='nearest')
\end{lstlisting}

Settings for display the matrix in the right way, in particular for the values to display on both the axis x and y, in this case every single months of the year.
\begin{lstlisting}
months = ["Jan","Feb","Mar","Apr","May","Jun",
	"Jul","Aug","Sep","Oct","Nov","Dec"]
x_pos = np.arange(len(months))
y_pos = np.arange(len(months))
pyplot.yticks(y_pos,months)
pyplot.xticks(x_pos,months)
\end{lstlisting}
Adding a title to the graphic that we are going to display and also a bar that works like a legend for the colors of the matrix, allowing the reader to better understand the values reported inside the matrix.
\begin{lstlisting}
pyplot.title("Correlation between different months")
pyplot.colorbar(cax)
\end{lstlisting}

There is the possibility to save the correlation matrix like an image and/or display it.
\begin{lstlisting}
saveFigure("_Months_Matrix.jpg")
pyplot.show()
\end{lstlisting}

\section{SIA section V: Single overview}
\label{SIA_section_V}
\textbf{Code implementation:}\\
create\_single\_overview() : this method will use the "Image" library for autogenerate a collage of the current input's graphics and save it like an overview image. The content of the params will basically decide how the "Current input overview image" will looks like.

It uses each single "current input overview image" of all the inputs and the "correlation matrix between all the inputs image" for combine them in a unique "total overview" and save it using the PDF format.

\begin{lstlisting}
listofimages=["CURRENT_INPUT_TOTAL_GRAPHIC",
            "CURRENT_INPUT_YEARS_MATRIX", 
            "CURRENT_INPUT_YEARS_GRAPHIC",
            "CURRENT_INPUT_MONTHS_MATRIX"]
            
create_single_overview(params1, listofimages)
create_single_overview(params2, listofimages)
\end{lstlisting}


The "create\_single\_overview" method has basically this structured, and then its configuration depends from the input data and from the preferences.
\begin{lstlisting}
def create_single_overview(cols, rows ,
			width, height, listofimages):
    thumbnail_width = width//cols
    thumbnail_height = height//rows
    size = thumbnail_width, thumbnail_height
    new_im = Image.new('RGB', (width, height))
    ims = []
    for p in listofimages:
        im = Image.open(p)
        im.thumbnail(size)
        ims.append(im)
    i = 0
    x = 0
    y = 0
    for col in range(cols):
        for row in range(rows):
            new_im.paste(ims[i], (x, y))
            i += 1
            y += thumbnail_height
        x += thumbnail_width
        y = 0
    new_im.save(SINGLE_OVERVIEW_IMAGE")
	new_im.show()
\end{lstlisting}
