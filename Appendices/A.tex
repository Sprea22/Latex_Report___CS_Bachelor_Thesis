\chap{SIA Implementation code}



\label{SIA_Implementation}

\section{SIA: Imported libraries}
\label{SIA_libraries}
The library "os" provides a way of using operating system dependent functionality.
\begin{lstlisting}
import os
\end{lstlisting}

The library "sys" is useful for test and execute the program, mainly because it allows to input directly via shell terminal.
\begin{lstlisting}
import sys
\end{lstlisting}

The "pandas" library will be very useful for read the data from CSV dataset and setup up the plot about it.
\begin{lstlisting}
import pandas as pd
\end{lstlisting}

The "numpy" library is the fundamental package for scientific computing with Python. It's used for mathematic purposes, such as calculating the correlation coefficient between two series.
\begin{lstlisting}
import numpy as np
\end{lstlisting}
 
The "pyplot" library it's used for basic graphic displaying and customization, easy to use but very efficient.
\begin{lstlisting}
import matplotlib.pyplot as pyplot
\end{lstlisting}

The library "PIL" supports many file formats, and provides powerful image processing and graphics capabilities.
\begin{lstlisting}
from PIL import Image
\end{lstlisting}

\section{SIA: Implemented methods}
This 'pyplot' style configuration allows to customize the graphic design. In this case was used a popular plotting package called 'ggplot'.
\begin{lstlisting}
pyplot.style.use('ggplot')
\end{lstlisting}

The two methods reported below here were used for calculating the trend line angular coefficients and the normalized one. In order to reach this goal is used the python library "numpy".
\begin{lstlisting}
def trendline(x, y, col):
	z = np.polyfit(x, y, 1)
	p = np.poly1d(z)
	pyplot.plot(x,p(x), c=col)
	z2 = trendlineNorm(x, normalization(y))
	return z[0], z2

def trendlineNorm(x, y):
	z = np.polyfit(x, y, 1)
	return z[0]
\end{lstlisting}

The following method was used for normalize the input values, that means adjusting values measured on different scales to a notionally common scale , in this case (0,1).
\begin{lstlisting}
def normalization(values):
	column = list(float(a) for a in range(0, 0))
	val = np.array(values)
	val.astype(float)
	column = val / val.max()
	return column
\end{lstlisting}

The following code shows the code of the two methods that are allowing to save images and matrix values to the library "os". 
\begin{lstlisting}
def saveFigure(descr):
    script_dir = os.path.dirname(__file__)
    results_dir = os.path.join(script_dir, "Results/" + sys.argv[1] + "/" + sys.argv[2]+"/")
    if not os.path.isdir(results_dir):
        os.makedirs(results_dir)

def saveMatrix(corrRes, dest):
	mat = np.matrix(corrRes)
	dataframe = pd.DataFrame(data=mat.astype(float))
	dataframe.to_csv(dest, sep=',', header=False, float_format='%.2f', index=False)
\end{lstlisting}

The following code represents the method that was used for creating the overview image generated by the SIA system, that is basically a collage of all the generated graphics about a particular parameter of the current input dataset. In order to reach this goal has been strongly used the library "PIL", that allows image elaboration using Python, and the libary "OS", for saving the results.
\begin{lstlisting}
def create_single_overview(cols, rows, dest, width, height, listofimages):
    thumbnail_width = width//cols
    thumbnail_height = height//rows
    size = thumbnail_width, thumbnail_height
    new_im = Image.new('RGB', (width, height))
    ims = []
    for p in listofimages:
        im = Image.open(p)
        im.thumbnail(size)
        ims.append(im)
    i = 0
    x = 0
    y = 0
    for col in range(cols):
        for row in range(rows):
            new_im.paste(ims[i], (x, y))
            i += 1
            y += thumbnail_height
        x += thumbnail_width
        y = 0
    if dest==0:
    	script_dir = os.path.dirname(__file__)
    	results_dir = os.path.join(script_dir, "Results/" + sys.argv[1]+"/"+sys.argv[2]+"/")
    	if not os.path.isdir(results_dir):
    		os.makedirs(results_dir)
        new_im.save(results_dir+"/"+ sys.argv[1] +"_"+sys.argv[2]+"_Graphics_Overview.jpg")
        new_im.show()
    if dest==1:
    	script_dir2 = os.path.dirname(__file__)
    	results_dir2 = os.path.join(script_dir2, "Results/" + sys.argv[1]+"/Total_Evidences/Single_Inputs")
    	if not os.path.isdir(results_dir2):
    		os.makedirs(results_dir2)
        new_im.save(results_dir2+"/"+ sys.argv[1] +"_"+sys.argv[2]+"_Overview.jpg")
\end{lstlisting}



\section{SIA section I: Total graphic for all the years}
\label{SIA_section_I}
\textbf{Code implementation:}\\
During this section of the code was used "pandas" library for read the dataset.
\begin{lstlisting}
series1 = pd.read_csv("Datasets/" + sys.argv[1]+".csv", usecols=[1,sys.argv[2]])
\end{lstlisting}

Then using the "pyplot" library has been possible to setup the plot of the input data.
\begin{lstlisting}
series1.plot(color="blue", linewidth=1.5)
\end{lstlisting}


There are some settings about the axis x just to display the data in the right format, are easy to change and to costume.
\begin{lstlisting}
years = []
j = 0
for i in range(len(yearInput)):
    if j==11:
        years.append(yearInput.values[i][0])
        j=0
    else:
        j=j+1 
x = range(0, len(yearInput.values))
pyplot.xticks(np.arange(min(x), max(x)+1, 12.0), years)
pyplot.title(sys.argv[1] + "\n" + sys.argv[2]+ ": Total graphic")
\end{lstlisting}

Once set up the plot of the current data, the next step was to display the trendline of the current graphic.

At this point the current data values have been read again and passed to the method just implemented above for calculating the trendline.
\begin{lstlisting}
series1 = pd.read_csv("Datasets/" + sys.argv[1]+".csv", usecols=[sys.argv[2]], squeeze=True)
z1, z2 = trendline(x, series1.values.astype(float), "red")
saveFigure("_Total.jpg")
results_dir = "Results/" + sys.argv[1]+"/"+sys.argv[2]+"/"+sys.argv[1]+"_"+sys.argv[2]+"_AngCoeff.csv"
with open(results_dir, "w") as text_file:
	text_file.write("," + sys.argv[1] + "-" + sys.argv[2]+"\n")
	text_file.write("," + "Ang_Coeff " + "," + str(z1)+"\n")
	text_file.write("," + "Norma_Ang_Coeff " + "," + str(z2)+"\n")
\end{lstlisting}

\section{SIA section II: Single graphics for each year}
\label{SIA_section_II}
\textbf{Code implementation:}\\
During this section of the code was used "pandas" library for read the dataset.
\begin{lstlisting}
series2 = pd.read_csv("Datasets/" + sys.argv[1]+".csv", index_col=['month'], usecols=[0,1,sys.argv[2]])
\end{lstlisting}

Some initialization of variables that are going to be useful.
\begin{lstlisting}
fig2 = pyplot.figure()
ax = fig2.add_subplot(111)
months = ["Jan","Feb","Mar","Apr","May","Jun","Jul","Aug","Sep","Oct","Nov","Dec"]
x_pos = np.arange(len(months))
\end{lstlisting}

The following code allows the system to split the values and display them in the right way: that means that are going to be split for each single year and then plotted on the same graphic.
\begin{lstlisting}
tempValues = []
j = 0
for i in range(len(series2.values)):
	if j in range(12):
		tempValues.append(series2.values[i][1])
		j = j + 1
		if(i == len(series2.values)-1):
			pyplot.plot(x_pos, tempValues, linewidth=2, alpha=0.8, label = int(series2.values[i-1][0]))
	else:
		pyplot.plot(x_pos, tempValues, linewidth=2, alpha=0.8, label = int(series2.values[i-1][0]))
		tempValues = []
		tempValues.append(series2.values[i][1])
		j = 1
\end{lstlisting}

These are some personalization settings that could be easily changed as you want.
\begin{lstlisting}
ax.legend(loc=4, ncol=1, fancybox=True, shadow=True)
pyplot.xticks(x_pos,months)
pyplot.xlim(0,11)
pyplot.title(sys.argv[1] + "\n" + sys.argv[2]+ ": Single year's graphic")
pyplot.tight_layout()
\end{lstlisting}

There is the possibility to save the graphic like an image and/or display it.
\begin{lstlisting}
saveFigure("_Years.jpg")
\end{lstlisting}


\section{SIA section III: Correlation matrix between years}
\label{SIA_section_III}
\textbf{Code implementation:}\\
During this section of the code was used "pandas" library for read the dataset.
\begin{lstlisting}
series3 = pd.read_csv("Datasets/" + sys.argv[1]+".csv", index_col=['month'], usecols=[0,1,sys.argv[2]])
\end{lstlisting}

\begin{lstlisting}
corr = []
tempValues = []
j = 0
# Collecting the correct values to elaborate.
for i in range(len(series3.values)+1):
	if j in range(12):
		tempValues.append(series3.values[i][1])
		j = j + 1
	else:
		corr.append(tempValues)
		tempValues = []
		if i in range(len(yearInput)):
			tempValues.append(series3.values[i][1])
			j = 1
\end{lstlisting}

With the library "numpy" is possible to calculate the correlation coefficients between all the variables in the series just read.
\begin{lstlisting}
corrRes = np.corrcoef(corr)
\end{lstlisting}

Setup the figure that will display the correlation matrix using the library "pypot".
\begin{lstlisting}
fig3 = pyplot.figure()
ax = fig3.add_subplot(111)
\end{lstlisting}

Creating the correlation matrix using the already calculated correlation coefficients.
\begin{lstlisting}
cax = ax.matshow(corrRes, interpolation='nearest')
\end{lstlisting}

Settings for display the matrix in the right way, in particular for the values to display on both the axis x and y, in this case every single year from 2005 to 2016
\begin{lstlisting}
pyplot.title(sys.argv[1] + "\n" + sys.argv[2]+ ": Correlation between different years")
x_pos = np.arange(yearsLen)
y_pos = np.arange(yearsLen)
pyplot.yticks(y_pos,years)
pyplot.xticks(x_pos,years)
pyplot.colorbar(cax)
\end{lstlisting}
\newpage
Adding a title to the graphic that we are going to display and also a bar that works like a legend for the colors of the matrix, allowing the reader to better understand the values reported inside the matrix.
\begin{lstlisting}
pyplot.tight_layout()
saveFigure("_years_Matrix.jpg")
saveMatrix(corrRes, "Results/"+sys.argv[1]+"/"+sys.argv[2]+"/"+sys.argv[1]+"_"+sys.argv[2]+"_years_CorrCoeff.csv")

\end{lstlisting}


\section{SIA section IV: Correlation matrix between months}
\label{SIA_section_IV}
\textbf{Code implementation:}\\
During this section of the code was used "pandas" library for read the dataset.
\begin{lstlisting}
series4 = pd.read_csv("Datasets/" + sys.argv[1]+".csv", usecols=[0,1,sys.argv[2]])
\end{lstlisting}

\begin{lstlisting}
corr = []
for month, year in series4.groupby(["month"], sort=False):
	corr.append(year[sys.argv[2]].values)
corrRes = np.corrcoef(corr)
\end{lstlisting}

Setup the figure that will display the correlation matrix using the library "pypot".
\begin{lstlisting}
fig4 = pyplot.figure()
ax = fig4.add_subplot(111)
\end{lstlisting}

Creating the correlation matrix using the already calculated correlation coefficents.
\begin{lstlisting}
cax = ax.matshow(test, interpolation='nearest')
\end{lstlisting}

Settings for display the matrix in the right way, in particular for the values to display on both the axis x and y, in this case every single months of the year.
\begin{lstlisting}

months = ["Jan","Feb","Mar","Apr","May","Jun","Jul","Aug","Sep","Oct","Nov","Dec"]
x_pos = np.arange(len(months))
y_pos = np.arange(len(months))
pyplot.yticks(y_pos,months)
pyplot.xticks(x_pos,months)
\end{lstlisting}
Adding a title to the graphic that we are going to display and also a bar that works like a legend for the colors of the matrix, allowing the reader to better understand the values reported inside the matrix.
\begin{lstlisting}
pyplot.title(sys.argv[1] + "\n" + sys.argv[2]+ ": Correlation between different months")
pyplot.colorbar(cax)
\end{lstlisting}

There is the possibility to save the correlation matrix like an image and/or display it.
\begin{lstlisting}
pyplot.tight_layout()
saveFigure("_months_Matrix.jpg")
saveMatrix(corrRes, "Results/"+sys.argv[1]+"/"+sys.argv[2]+"/"+sys.argv[1]+"_"+sys.argv[2]+"_months_CorrCoeff.csv")

\end{lstlisting}

\section{SIA section V: Single overview}
\label{SIA_section_V}
\textbf{Code implementation:}\\
create\_single\_overview() : this method will use the "Image" library in order to generates a collage of the current input's graphics and save it like an overview image. The content of the parameters will basically decide how the "Current input overview image" will looks like.

It uses each single "current input overview image" of all the inputs and the "correlation matrix between all the inputs image" for combine them in a unique "total overview" and save it using the PDF format.

\begin{lstlisting}
listofimages=["Results/" + sys.argv[1]+"/"+sys.argv[2]+"/"+sys.argv[1]+"_"+sys.argv[2]+"_Total.jpg",
            "Results/" + sys.argv[1]+"/"+sys.argv[2]+"/"+sys.argv[1]+"_"+sys.argv[2]+"_years_Matrix.jpg", 
            "Results/" + sys.argv[1]+"/"+sys.argv[2]+"/"+sys.argv[1]+"_"+sys.argv[2]+"_years.jpg",
            "Results/" + sys.argv[1]+"/"+sys.argv[2]+"/"+sys.argv[1]+"_"+sys.argv[2]+"_months_Matrix.jpg"]

create_single_overview(4, 1, 1, 3200, 600, listofimages)
create_single_overview(2, 2, 0, 1600, 1200, listofimages)
\end{lstlisting}
