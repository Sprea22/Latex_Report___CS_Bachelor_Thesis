\chap{MIA Implementation code}
\label{MIA_Implementation}
\section{MIA: Imported libraries}
\label{MIA_Libraries}
The "pandas" library will be very useful for read the data from CSV dataset and setup the plot abut it.
\begin{lstlisting}
import pandas as pd
\end{lstlisting}

The "numpy" library it's used for mathematic purpose, such as calculating the correlation coefficent between two series.
\begin{lstlisting}
import numpy as np
\end{lstlisting}

 
The "pyplot" library it's used for basic graphic displaying and customization, easy to use but very efficent.
\begin{lstlisting}
import matplotlib.pyplot as pyplot
\end{lstlisting}

\section{MIA: Implementation}
\textbf{Code implementation:}\\
First of all, we are going to use the "pandas" library for read the dataset.
\begin{lstlisting}
series3 = pd.read_csv("TOTAL_DATASET_DIRECTORY", 
	index_col=['Input'], header=0)
\end{lstlisting}

Then with the library "numpy" is possible to calculate the correlation coefficents between all the variables just read above.
\begin{lstlisting}
test = np.corrcoef(series3.values)
\end{lstlisting}

Setup the figure that will display the correlation matrix using the library "pyplot".
\begin{lstlisting}
fig2 = pyplot.figure()
ax = fig2.add_subplot(111)
\end{lstlisting}

Creating the correlationg matrix using the already calculated correlation coefficents.
\begin{lstlisting}
cax = ax.matshow(test, interpolation='nearest')
\end{lstlisting}

Settings for display the matrix in the right way, in particular for the values to display on both the axis x and y, in this case every single input.
\begin{lstlisting}
inputs = ["Cages", "Feed", "Number", "Restock",
	"Local", "Withdr", "Biomass", "Price"]
x_pos = np.arange(len(inputs))
y_pos = np.arange(len(inputs))
pyplot.yticks(y_pos,inputs)
pyplot.xticks(x_pos,inputs)
\end{lstlisting}

Adding a title to the graphic that we are going to display and also a ba that works like a legend for the colors of the matrix, allowing the reader to better understand the values reported inside the matrix.
\begin{lstlisting}
pyplot.title("Correlation between different inputs 
		about data from 2005 to 2016")
pyplot.colorbar(cax)
\end{lstlisting}

In the end, using again the library "pyplot", there is the possibility to save the correlation matrix graphic like an image and/or display it.
\begin{lstlisting}
pyplot.savefig("OUTPUT_DIRECTORY")
\end{lstlisting}

\begin{lstlisting}
series = pd.read_csv("TRENDLINES_VALUES_DOCUMENT",
	 header=0, usecols=["Norm Ang Coeffs"])
series.plot(kind="barh")
pyplot.savefig("OUTPUT_DIRECTORY")

create_total_overview()
\end{lstlisting}

\begin{lstlisting}
pyplot.show()
\end{lstlisting}

