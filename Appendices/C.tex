\chapter{Norway's Map System Implementation Code}
\label{Map_System}
\section{Map System: Imported libraries}
\label{Map_libraries}
The library report below here have already been used during the previous phase of the implementation work, and is possible to check out their utilities here: [\ref{SIA_libraries}]
\begin{lstlisting}
import os
import sys
import matplotlib
import pandas as pd
import matplotlib.pyplot as plt
\end{lstlisting}

The library "cartopy" provides cartographic tools for Python, in particular "crs" (Coordinate Reference Systems) that is the very core of 'cartopy' since allow to use a list of several projections with the same interface. Furthermore, the class "shapereader" provides an interface for accessing the contents of a shapefile.
\begin{lstlisting}
import cartopy.crs as ccrs
import cartopy.io.shapereader as shpreader
\end{lstlisting}

The library "matplolib" provides with the module "cm" a large set of colormaps and with the "patches" module it allows to draw some geometry figures.
\begin{lstlisting}
import matplotlib.cm as cmx
import matplotlib.patches as mpatches
\end{lstlisting}

\section{Norwegian map implementation}
\label{Map_system_implementation}
The following implemented method allows to give a specific shapely geometries to the axes. The method get this particular parameter with "shapeInput". Furthermore, "labelInput" is in this case the name of the current Norwegian county and the "colorInput" the specific color which will be used for display the geometry.
\begin{lstlisting}
def add_geom(axes, shapeInput, labelInput, colorInput):
	axes.add_geometries(shapeInput, ccrs.Robinson(), edgecolor='black',label = labelInput, facecolor=colorInput, alpha=0.8)
	return mpatches.Rectangle((0, 0), 1, 1, facecolor=colorInput)
\end{lstlisting}

The shapefile "NOR\_adm1.shp"\footnote{Shapefile about Norway's territory download link: \\ \url{http://biogeo.ucdavis.edu/data/gadm2.8/shp/NOR\_adm\_shp.zip}} contains the needed data for display each single Norwegian county. The variable "NOR\_shapes" allows to access in a easier way to this values, since it's a list which each position corresponds to a specific county shape.
\begin{lstlisting}
fname = 'Datasets/NOR/NOR_adm1.shp'
NOR_shapes = list(shpreader.Reader(fname).geometries())
\end{lstlisting}

How reported in the implementation part during this part of the work was created a specific dataset "countiesAverages". [\ref{Map_displaying}] The system creates a list that contains the value of the parameter "dataInput" for each county.
\begin{lstlisting}
dataInput = sys.argv[1]
inputSeries = pd.read_csv("Datasets/countiesAverages.csv")
inputValues = [inputSeries[dataInput][0], inputSeries[dataInput][1], inputSeries[dataInput][2], inputSeries[dataInput][3], inputSeries[dataInput][4], inputSeries[dataInput][5], inputSeries[dataInput][6], inputSeries[dataInput][7], inputSeries[dataInput][8]]
\end{lstlisting}

In the following code the class "Normalize" is used for normalize the data input into [vmin , vmax] interval, that are respectively max and min for input values. 
\begin{lstlisting}
minValues = min(inputValues)
maxValues = max(inputValues)
cNorm = matplotlib.colors.Normalize(vmin=minValues, vmax=maxValues)
\end{lstlisting}

\newpage

The "colMap" variable contains a specific range of colors and it will be used together with the normalized range of values by the class "ScalarMappable" that returns RGBA colors, that with the function "colorbar" are used for generate a legend about the values and the colors.
\begin{lstlisting}
colMap='bwr'
cm = plt.get_cmap(colMap)
scalarMap = cmx.ScalarMappable(norm=cNorm, cmap=cm)	
col = scalarMap.to_rgba(inputValues)	
scalarMap.set_array(inputValues)
plt.colorbar(scalarMap,label='Input Value')
\end{lstlisting}

Once the system has the values and the relative colors, it uses the "add\_geom" method in order to display the geometric shape of each country with the related color and label. 
\begin{lstlisting}
ax = plt.axes(projection=ccrs.Robinson())
ax.coastlines(resolution='10m')
ax.set_extent([4, 32, 57, 72], ccrs.Robinson())	
norway = add_geom(ax, NOR_shapes, "Norway", "gray")
finnmark = add_geom(ax, NOR_shapes[4], "Finnmark", col[0])
troms = add_geom(ax, NOR_shapes[16], "Troms", col[1])
nordland = add_geom(ax, NOR_shapes[9], "Nordland", col[2])
nord_trondelag = add_geom(ax, NOR_shapes[8], "Nord Trondelag", col[3])
sor_trondelag = add_geom(ax, NOR_shapes[13], "Sor Trondelag", col[4])
more_og_romsdal = add_geom(ax, NOR_shapes[7], "More og Romsdal", col[5])
sogn_og_fjordane = add_geom(ax, NOR_shapes[14], "Sogn og Fjordane", col[6])
hordaland = add_geom(ax, NOR_shapes[6], "Hordaland", col[7])
rogaland_og_agder = add_geom(ax,NOR_shapes[2], "Rogaland og Agder",col[8])
rogaland_og_agder = add_geom(ax,NOR_shapes[12], "Rogaland og Agder",col[8])
rogaland_og_agder = add_geom(ax,NOR_shapes[17], "Rogaland og Agder",col[8])
\end{lstlisting}

Final settings for modify the title of the graphic and display a legend with the correct labels. \\
The "manager.resize(*manager.window.maxsize())" function allows to maximize to full screen the displayed graphic with "plt.show".
\begin{lstlisting}
plt.title('Norway - '+sys.argv[1] , fontsize=35)
labels = ['Finnmark', 'Troms', 'Nordland','Nord Trondelag', 'Sor Trondelag', 'More og Romsdal',	'Sogn og Fjordane', 'Hordaland', 	'Rogaland og Agder','Other counties', ]
plt.legend([finnmark, troms, nordland, nord_trondelag, sor_trondelag, 
	more_og_romsdal, sogn_og_fjordane, hordaland, rogaland_og_agder, norway], 
	labels, loc='lower right', fancybox=True)
manager = plt.get_current_fig_manager()
manager.resize(*manager.window.maxsize())
plt.show()


\end{lstlisting}