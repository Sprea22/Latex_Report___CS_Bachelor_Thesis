\chap{Implementation}

\lstset{
language=Python,
backgroundcolor = \color{Ivory},
basicstyle=\ttfamily,
otherkeywords={self},             
keywordstyle=\ttfamily\color{blue!90!black},
keywords=[2]{True,False},
keywords=[3]{ttk},
keywordstyle={[2]\ttfamily\color{yellow!80!orange}},
keywordstyle={[3]\ttfamily\color{red!80!orange}},
emph={MyClass,__init__},          
emphstyle=\ttfamily\color{red!80!black},    
stringstyle=\color{green!80!black},
showstringspaces=false            
}

\hypersetup{
    colorlinks=true,
    linkcolor=blue,
    filecolor=magenta,      
    urlcolor=blue,
}


Total implementation link for Experiment 1 : \\
\url{https://github.com/Sprea22/Data_Analyzer_Python}

Total implementation link for Experiment 2 : \\
\url{}

(INSTRUCTION FOR DOWNLOAD AND USE FROM GITHUB)

\newpage
\section{1st Phase: Data collection}
During this phase is important to be sure to have all the data that we are going to need. During this particular implementation we need the 7 input dataset written above in the "Dataset structure" section, and below here you can find the link of the website where you can download all the needed datasets for this example.

Datasets downloads website:\\ 
\url{http://www.fiskeridir.no/Akvakultur/Statistikk-akvakultur/Biomassestatistikk}

The datasets that you can download on this website are in a different format than the one we will need, but for better understand the meaning of the data are useful since are in XLSX format, splitted in clear tables and well commented (only in norwegian).

\subsection{Data sources}
\subsection{Data description and validation}

\newpage
\section{2nd Phase: Dataset creation}
\subsection{Processing the data}

Once we have all the "row" data we can start to setup the dataset that we will need for this system. 
First of all we will use the format CSV for the new datasets instead of XLSX.
Then we need to create three different "standards" of each input dataset. The reason of this requirement is that using the python library for some kind of analysis is easier to read the data in a particular format instead of reading the data always in the normal CSV standard and then edit and handle it later through the code.
Following are reported the needed standards.

\begin{itemize}
\item Normal CSV standard (Standard\_N)
\begin{table}[ht] 
    \centering 
    \begin{tabular}{ | l | l |}
        \hline
       	Month & Input\_Name\\ \hline
        January\_2005 & Value1 \\ \hline
        February\_2005 & Value2\\ \hline
        March\_2005 & Value3\\ \hline
        ... & ...\\ \hline
        November\_2016 & Value143\\ \hline
        December\_2016 & Value144\\ \hline
    \end{tabular}
    \caption[Dataset Normal Standard]{Dataset Normal Standard}
    \label{table: Dataset_Normal_Standard} 
\end{table}    

\item Month standard (Standard\_M)
\begin{table}[ht] 
    \centering 
    \begin{tabular}{ | l | l | l | l | l | l |}
        \hline
       	Month & 2005 & 2006 & 2007 & ... & 2016\\ \hline
        January & value1 & value13 & value25 & ... & value133\\ \hline
        February & value2 & value14 & value26 & ... & value134\\ \hline
        March & value3 & value15 & value27 & ... & value135\\ \hline
        ... & ... & ... & ... & ... & ...\\ \hline
        December & value12 & value24 & value36 & ... & value144\\ \hline
    \end{tabular}
    \caption[Dataset Month Standard]{Dataset Month Standard}
    \label{table: Dataset_Month_Standard} 
\end{table}  

\item Year standard (Standard\_Y)
\begin{table}[ht] 
    \centering 
    \begin{tabular}{ | l | l | l | l | l | l |}
        \hline
       	Year & January & February & March & ... & December\\ \hline
        2005 & value1 & value2 & value3 & ... & value12\\ \hline
        2006 & value13 & value14 & value15 & ... & value24\\ \hline
        2007 & value25 & value26 & value27 & ... & value36\\ \hline
        ... & ... & ... & ... & ... & ...\\ \hline
        2016 & value133 & value134 & value135 & ... & value144\\ \hline
    \end{tabular}
    \caption[Dataset Year Standard]{Dataset Year Standard}
    \label{table: Dataset_Year_Standard} 
\end{table}  
\end{itemize}

The last step of this phase is to setup the final dataset, creating for each data input three different CSV file that are following the three standard reported above.\\
The following strucutre show how our final dataset will looks like once we apply the standards on each one of our data inputs.
\begin{enumerate}
\item Cages
	\begin{itemize}
		\item Cages.csv : Contains the data about "Cages" following the Stand\_N
		\item Cages\_M.csv : Contains the data about "Cages" following the Stand\_M
		\item Cages\_Y.csv : Contains the data about "Cages" following the Stand\_Y
	\end{itemize}
\item Localities
	\begin{itemize}
		\item Localities.csv : Contains the data about "Localities" following the Stand\_N
		\item Localities\_M.csv : Contains the data about "Localities" following the Stand\_M
		\item Localities\_Y.csv : Contains the data about "Localities" following the Stand\_Y
	\end{itemize}
Continue and do the same thing for all the other data input
\item Monthly\_salmon\_price 
\item Salmon\_biomass\_end\_month
\item Salmon\_consumption\_of\_feed
\item Salmon\_number\_end\_month
\item Salmon\_restock
\item Salmon\_withdrawals
\end{enumerate}

And in the end we will create also a total dataset of all the different inputs, that we will call it Total\_Dataset.csv and looks like:
\begin{table}[ht] 
    \label{table:Total_Dataset} 
    \centering 
    \begin{tabular}{ | l | l | l | l | l | l | l |}
        \hline
       	Input & January\_2005 & February\_2005 & March\_2005 & ... & November\_2016 & December\_2016\\ \hline
        Input1 & value1.1 & value1.2 & value1.3 & ... & value1.143 & value1.144\\ \hline
        Input2 & value2.1 & value2.2 & value2.3 & ... & value2.143 & value2.144\\ \hline
        Input3 & value3.1 & value3.2 & value3.3 & ... & value3.143 & value3.144\\ \hline
        ... & ... & ... & ... & ... & ...\\ \hline
        Input7 & value7.1 & value7.2 & value7.3 & ... & value7.143 & value7.144\\ \hline
        Input8 & value8.1 & value8.2 & value8.3 & ... & value8.143 & value8.144\\ \hline
    \end{tabular}
    \caption[Total Dataset]{Total Dataset}
\end{table}  
\newpage


\part{Experiment 1: Data analysis system}
\section{3nd Phase: Analyzer system implementation}
Total implementation link for Experiment 1 : \\
\url{https://github.com/Sprea22/Data_Analyzer_Python}

During this part the main purpose is to analyse the whole dataset in order to find some kind of useful informations later on. \\
The Python system that we are going to implement is mainly used for a generic analysis of the data under from different point of views.\\
The output of this phase will basically be for each single data input:
\begin{itemize}
\item Total graphic of the input data from 2005 to 2016.
\item Graphic of the input data for each single year from 2005 to 2016.
\item Correlation matrix between different months of the same input.
\item Correlation matrix between different years of the same input.
\end{itemize}


It's important to remind that this phase can be implemented in different ways and with different programming language, in this case the programming language choosen is Python, so be sure to have installed all the necessary for compile and execute Python code on your platform.\\
(PYTHON REQUIREMENTS)\\
During this experiment we are going to implement a system that is basically divided in two subsystems, that are:
\begin{itemize}
\item Single Input Analyzer (SIA): Used for analyze a single data input.
\item Multiple Inputs Analyzer (TIA): Used for analyze multiple data inputs.
\end{itemize}
\newpage


\newpage
\subsection{Single Input Analyzer}
\begin{itemize}
\item SIA imported libraries.
\item SIA part I: Generate and display a graphic about current input with total data from 2005 to 2016.
\item SIA part II: Generate and display a graphic about current input for each year from 2005 to 2016.
\item SIA part III: Generate and display a graphic that contains the correlation matrix between each single year from 2005 to 2016 of the current input.
\item SIA part IV: Generate and display a graphic that contains the correlation matrix between each single months of the year of the current input.
\item SIA part V: Generate and display a single overview image for the current input and update the total overview PDF.
\end{itemize}

\newpage
\subsubsection{SIA: Imported libraries}
The "pandas" library will be very useful for read the data from CSV dataset and setup the plot abut it.
\begin{lstlisting}
import pandas as pd
\end{lstlisting}

The "numpy" library it's used for mathematic purpose, such as calculating the correlation coefficent between two series.
\begin{lstlisting}
import numpy as np
\end{lstlisting}
 
The "pyplot" library it's used for basic graphic displaying and customization, easy to use but very efficent.
\begin{lstlisting}
import matplotlib.pyplot as pyplot
\end{lstlisting}

Also the library "sys" would be very useful for test and execute the program, mainly because it allows to input directly from terminal.
\begin{lstlisting}
import sys
\end{lstlisting}

The library "PIL" supports many file formats, and provides powerful image processing and graphics capabilities.
\begin{lstlisting}
from PIL import Image
\end{lstlisting}

The library "fpdf" allows to generate and use PDF file.
\begin{lstlisting}
from fpdf import FPDF
\end{lstlisting}
\newpage
\subsubsection{SIA section I: Total graphic for all the years}
\textbf{Goal:}\\
Generate and display the total graphic about current input from 2005 to 2016.

\textbf{Requirements:}\\
- Input dataset: CSV format following the Stand\_N \\
- Data content: 144 values, 1 value for each month from 2005 to 2016

\textbf{Code implementation:}\\
During this section of the code we will use the "pandas" library for read the dataset.
\begin{lstlisting}
series = pd.read_csv("DATASET_DIRECTORY", header=0)
\end{lstlisting}

Then using the "pyplot" library we can setup the plot of the input data.
\begin{lstlisting}
series.plot(color="blue", linewidth=1.5)
\end{lstlisting}


Thera are some settings about the axis x just to display the data in the right format, are easy to change and to costume.
\begin{lstlisting}
years = ["2005","2006","2007","2008","2009","2010",
	"2011","2012","2013","2014","2015","2016"]
x = range(144)
pyplot.xticks(np.arange(min(x), max(x)+1, 12.0), years)
pyplot.title(Total graphic from 2005 to 2016")
\end{lstlisting}

There is the possibility to save the graphic like an image and/or display it.
\begin{lstlisting}
pyplot.savefig("OUTPUT_DIRECTORY", format="jpg")
pyplot.show()
\end{lstlisting}




\begin{minipage}{0.5\textwidth}
\textbf{Results:} \\
With this first part of the code we are able to display and save the basic graphic about the current input from 2005 to 2016, that looks like this example:
\end{minipage} \hfill
\begin{minipage}{0.45\textwidth}
\begin{figure}[H]
\includegraphics[width=0.9\textwidth]{Files/Cages_Total.jpg}
\caption{Total graphic about current input with total data from 2005 to 2016.}
\end{figure}
\end{minipage}


\newpage
\subsubsection{SIA section II: Single graphics for each year}

\textbf{Goal:}\\
Generate and display graphics about current input for each year from 2005 to 2016

\textbf{Requirements:}\\
- Input dataset: CSV format following the Stand\_M \\
- Data content: 144 values, 1 value for each month from 2005 to 2016

\textbf{Code implementation:}\\
During this section of the code we will use the "pandas" library for read the dataset.
\begin{lstlisting}
series2 = pd.read_csv("DATASET_DIRECTORY", 
	index_col=['Month'], 
	header=0, usecols=[0,1,2,3,4,5,6,7,8,9,10,11,12])
\end{lstlisting}

Then using the "pyplot" library we can setup the plot of the input data.
\begin{lstlisting}
series2.plot()
\end{lstlisting}

Adding the title at the graphic that we are going to display.
\begin{lstlisting}
pyplot.title("Single year's graphic from 2005 to 2016")
\end{lstlisting}

There is the possibility to save the graphic like an image and/or display it.
\begin{lstlisting}
pyplot.savefig("OUTPUT_DIRECTORY", format="jpg")
pyplot.show()
\end{lstlisting}


\begin{minipage}{0.5\textwidth}
\textbf{Results:} \\
With this second part of the code we are able to display and save the graphics about the current input for each single year from 2005 to 2016, that looks like this example:
\end{minipage} \hfill
\begin{minipage}{0.45\textwidth}
\begin{figure}[H]
    \includegraphics[width=0.9\textwidth]{Files/Cages_Years.jpg}
    \caption{Graphics for each single year of the input data from 2005 to 2016}
\end{figure}
\end{minipage}



\newpage
\subsubsection{SIA section III: Correlation matrix between years}

\textbf{Goal:}\\
Generate and display the correlation matrix about current input between each single year from 2005 to 2016

\textbf{Requirements:}\\
- Input dataset: CSV format following the Stand\_Y \\
- Data content: 144 values, 1 value for each month from 2005 to 2016

\textbf{Code implementation:}\\
During this section of the code we will use the "pandas" library for read the dataset.
\begin{lstlisting}
series3 = pd.read_csv("DATASET_DIRECTORY",
	 header=0, usecols=[1,2,3,4,5,6,7,8,9,10,11,12])
\end{lstlisting}

With the library "numpy" is possible to calculate the correlation coefficents between all the variables in the series just read.
\begin{lstlisting}
test = np.corrcoef(series3.values)
\end{lstlisting}

Setup the figure that will display the correlation matrix using the library "pypot".
\begin{lstlisting}
fig2 = pyplot.figure()
ax = fig2.add_subplot(111)
\end{lstlisting}

Creating the correlation matrix using the already calculated correlation coefficents.
\begin{lstlisting}
cax = ax.matshow(test, interpolation='nearest')
\end{lstlisting}

Settings for display the matrix in the right way, in particular for the values to display on both the axis x and y, in this case every single year from 2005 to 2016
\begin{lstlisting}
years = ["2005","2006","2007","2008","2009","2010",
	"2011","2012","2013","2014","2015","2016"]
x_pos = np.arange(len(years))
y_pos = np.arange(len(years))
pyplot.yticks(y_pos,years)
pyplot.xticks(x_pos,years)
\end{lstlisting}
\newpage
Adding a title to the graphic that we are going to display and also a bar that works like a legend for the colors of the matrix, allowing the reader to better understand the values reported inside the matrix.
\begin{lstlisting}
pyplot.title("Correlation between different years")
pyplot.colorbar(cax)
\end{lstlisting}

There is the possibility to save the correlation matrix like an image and/or display it.
\begin{lstlisting}
pyplot.savefig("OUTPUT_DIRECTORY", format="jpg")
pyplot.show()
\end{lstlisting}

\begin{minipage}{0.5\textwidth}
\textbf{Results:} \\
With this second part of the code we are able to display and save the correlation matrix about current input between each single year from 2005 to 2016, that looks like this example:
\end{minipage} \hfill
\begin{minipage}{0.45\textwidth}
\begin{figure}[H]
    \includegraphics[width=0.9\textwidth]{Files/Cages_Months_Matrix.jpg}
    \caption{Correlation matrix between different months of the same input}
\end{figure}
\end{minipage}



\newpage
\subsubsection{SIA section IV: Correlation matrix between months}

\textbf{Goal:}\\
Correlation matrix about current input between each single month from 2005 to 2016

\textbf{Requirements:}\\
- Input dataset: CSV format following the Stand\_M \\
- Data content: 144 values, 1 value for each month from 2005 to 2016

\textbf{Code implementation:}\\
During this section of the code we will use the "pandas" library for read the dataset.
\begin{lstlisting}
series4 = pd.read_csv("DATASET_DIRECTORY", header=0, 
	usecols=[1,2,3,4,5,6,7,8,9,10,11,12])
\end{lstlisting}

With the library "numpy" is possible to calculate the correlation coefficents between all the variables in the series just read.
\begin{lstlisting}
test = np.corrcoef(series4.values)
\end{lstlisting}

Setup the figure that will display the correlation matrix using the library "pypot".
\begin{lstlisting}
fig2 = pyplot.figure()
ax = fig2.add_subplot(111)
\end{lstlisting}

Creating the correlation matrix using the already calculated correlation coefficents.
\begin{lstlisting}
cax = ax.matshow(test, interpolation='nearest')
\end{lstlisting}

Settings for display the matrix in the right way, in particular for the values to display on both the axis x and y, in this case every single months of the year.
\begin{lstlisting}
months = ["Jan","Feb","Mar","Apr","May","Jun",
	"Jul","Aug","Sep","Oct","Nov","Dec"]
x_pos = np.arange(len(months))
y_pos = np.arange(len(months))
pyplot.yticks(y_pos,months)
pyplot.xticks(x_pos,months)
\end{lstlisting}
\newpage
Adding a title to the graphic that we are going to display and also a bar that works like a legend for the colors of the matrix, allowing the reader to better understand the values reported inside the matrix.
\begin{lstlisting}
pyplot.title("Correlation between different months")
pyplot.colorbar(cax)
\end{lstlisting}

There is the possibility to save the correlation matrix like an image and/or display it.
\begin{lstlisting}
pyplot.savefig("OUTPUT_DIRECTORY", format="jpg")
pyplot.show()
\end{lstlisting}

\begin{minipage}{0.5\textwidth}
\textbf{Results:} \\
With this second part of the code we are able to display and save the correlation matrix about current input between each single month from 2005 to 2016, that looks like this example:
\end{minipage} \hfill
\begin{minipage}{0.45\textwidth}
\begin{figure}[H]
    \includegraphics[width=0.9\textwidth]{Files/Cages_Years_Matrix.jpg}
    \caption{Correlation matrix between different years of the same input}
\end{figure}
\end{minipage}

\newpage
\subsubsection{SIA section V: Single and Total overview}

\textbf{Goal:}\\
Generate and display a single overview image for the current input and update the total overview PDF.

\textbf{Requirements:}\\
- All the images that contain graphic about current input.

\textbf{Code implementation:}\\
The current code is basically composed from two methods, that are:
\begin{itemize}
\item create\_single\_overview() : this method will use the "Image" library for autogenerate a collage of the current input's graphics and save it like an overview image. The content of the params will basically decide how the "Current input overview image" will looks like.
\item create\_total\_overview() : this method it's created for update the "total overview pdf".
It uses each single "current input overview image" of all the inputs for combine them in a unique "total overview" and save it using the PDF format.
\end{itemize}
\begin{lstlisting}
listofimages=["CURRENT_INPUT_TOTAL_GRAPHIC",
            "CURRENT_INPUT_YEARS_MATRIX", 
            "CURRENT_INPUT_YEARS_GRAPHIC",
            "CURRENT_INPUT_MONTHS_MATRIX"]
           
create_single_overview(params1, listofimages)
create_single_overview(params2, listofimages)
create_total_overview()
\end{lstlisting}

\newpage
The "create\_total\_overview" method has basically this structured, and then its configuration depends from the input data and from the preferences.
\begin{lstlisting}
def create_total_overview():
    listof=["INPUT1_OVERVIEW_IMAGE",
			"INPUT2_OVERVIEW_IMAGE",
			"INPUT3_OVERVIEW_IMAGE",
			"INPUT4_OVERVIEW_IMAGE",
			"INPUT5_OVERVIEW_IMAGE",
			"INPUT6_OVERVIEW_IMAGE",
			"INPUT7_OVERVIEW_IMAGE",
			"INPUT8_OVERVIEW_IMAGE"]	
    pdf = FPDF(orientation = 'L')
    for image in listof:
        pdf.add_page()
        pdf.line(0,y-5,300,y-5)
        pdf.image(image,x,y,w,h)
    pdf.output("TOTAL_OVERVIEW_PDF", "F")
\end{lstlisting}

The "create\_single\_overview" method has basically this structured, and then its configuration depends from the input data and from the preferences.
\begin{lstlisting}
def create_single_overview(cols, rows ,
			width, height, listofimages):
    thumbnail_width = width//cols
    thumbnail_height = height//rows
    size = thumbnail_width, thumbnail_height
    new_im = Image.new('RGB', (width, height))
    ims = []
    for p in listofimages:
        im = Image.open(p)
        im.thumbnail(size)
        ims.append(im)
    i = 0
    x = 0
    y = 0
    for col in range(cols):
        for row in range(rows):
            new_im.paste(ims[i], (x, y))
            i += 1
            y += thumbnail_height
        x += thumbnail_width
        y = 0
    new_im.save(SINGLE_OVERVIEW_IMAGE")
	new_im.show()
\end{lstlisting}

\textbf{Results:}
\begin{figure}[H]
	\centering
    \includegraphics[width=1\textwidth]{Files/Cages_Overview.jpg}
    \caption{Example of "Single Input Overview Image"}
\end{figure}

\begin{figure}[H]
	\centering
    \includegraphics[width=1\textwidth]{Files/Total_Overview.jpg}
    \caption{Example of a single page of the "Total Overview PDF"}
\end{figure}





\newpage
\subsection{Multiple Inputs Analyzer}

\newpage
\subsubsection{MIA: Imported libraries}
The "pandas" library will be very useful for read the data from CSV dataset and setup the plot abut it.
\begin{lstlisting}
import pandas as pd
\end{lstlisting}

The "numpy" library it's used for mathematic purpose, such as calculating the correlation coefficent between two series.
\begin{lstlisting}
import numpy as np
\end{lstlisting}
 
The "pyplot" library it's used for basic graphic displaying and customization, easy to use but very efficent.
\begin{lstlisting}
import matplotlib.pyplot as pyplot
\end{lstlisting}

Also the library "sys" would be very useful for test and execute the program, mainly because it allows to input directly from terminal.
\begin{lstlisting}
import sys
\end{lstlisting}

\newpage
\subsubsection{MIA: Implementation}
\textbf{Goal:}\\
This analyzer is mainly used for show the correlation coefficent between the diffent inputs along the total period (from 2005 o 2016), that it will be important to have a general about which kind of relation there is between different inputs and how much strong it is.

\textbf{Requirements:}\\
- Input dataset: Total\_Dataset required, structure already reported here:
\hyperref[table: Total_Dataset]{Total Dataset}

\textbf{Code implementation:}\\
First of all, we are going to use the "pandas" library for read the dataset.
\begin{lstlisting}
series3 = pd.read_csv("TOTAL_DATASET_DIRECTORY", 
	index_col=['Input'], header=0)
\end{lstlisting}

Then with the library "numpy" is possible to calculate the correlation coefficents between all the variables just read above.
\begin{lstlisting}
test = np.corrcoef(series3.values)
\end{lstlisting}

Setup the figure that will display the correlation matrix using the library "pyplot".
\begin{lstlisting}
fig2 = pyplot.figure()
ax = fig2.add_subplot(111)
\end{lstlisting}

Creating the correlationg matrix using the already calculated correlation coefficents.
\begin{lstlisting}
cax = ax.matshow(test, interpolation='nearest')
\end{lstlisting}

Settings for display the matrix in the right way, in particular for the values to display on both the axis x and y, in this case every single input.
\begin{lstlisting}
inputs = ["Cages", "Feed", "Number", "Restock",
	"Local", "Withdr", "Biomass", "Price"]
x_pos = np.arange(len(inputs))
y_pos = np.arange(len(inputs))
pyplot.yticks(y_pos,inputs)
pyplot.xticks(x_pos,inputs)
\end{lstlisting}

Adding a title to the graphic that we are going to display and also a ba that works like a legend for the colors of the matrix, allowing the reader to better understand the values reported inside the matrix.
\begin{lstlisting}
pyplot.title("Correlation between different inputs 
		about data from 2005 to 2016")
pyplot.colorbar(cax)
\end{lstlisting}

In the end, using again the library "pyplot", there is the possibility to save the correlation matrix graphic like an image and/or display it.
\begin{lstlisting}
pyplot.savefig("OUTPUT_DIRECTORY")
pyplot.show()
\end{lstlisting}

\textbf{Results:} \\

\begin{figure}[H]
	\centering
    \includegraphics[width=0.75\textwidth]{Files/Total_Dataset_Years_Matrix.jpg}
    \caption{Correlation matrix between different inputs with data from 2005 to 2016}
\end{figure}


\newpage
\section{4th Phase: Data displaying}
\iffalse

\begin{figure}[h]
    \centering
    \begin{subfigure}[t]{0.5\textwidth}
        \centering
        \includegraphics[height=1.2in]{Files/Cages_Total.jpg}
        \caption{Total graphic of the input data from 2005 to 2016}
    \end{subfigure}%
    ~ 
    \begin{subfigure}[t]{0.5\textwidth}
        \centering
        \includegraphics[height=1.2in]{Files/Cages_Years.jpg}
        \caption{Graphics for each single year of the input data from 2005 to 2016}
    \end{subfigure}
    \caption{Data input trends graphics}
\end{figure}

\begin{figure}[h]

    \begin{subfigure}[t]{0.5\textwidth}
        \centering
        \includegraphics[height=1.2in]{Files/Cages_Months_Matrix.jpg}
        \caption{Correlation matrix between different months of the same input}
    \end{subfigure}%
    ~ 
    \begin{subfigure}[t]{0.5\textwidth}
        \centering
        \includegraphics[height=1.2in]{Files/Cages_Years_Matrix.jpg}
        \caption{Correlation matrix between different years of the same input}
    \end{subfigure}
    \caption{Data input correlations matrices}
\end{figure}

\begin{figure}[h]
    \centering
    \begin{subfigure}[t]{0.20\textwidth}
        \centering
        \includegraphics[height=1.2in]{Files/Cages_Total.jpg}

    \end{subfigure}%
    ~ 
    \begin{subfigure}[t]{0.20\textwidth}
        \centering
        \includegraphics[height=1.2in]{Files/Cages_Years.jpg}

    \end{subfigure}
    ~ 
    \begin{subfigure}[t]{0.20\textwidth}
        \centering
        \includegraphics[height=1.2in]{Files/Cages_Months_Matrix.jpg}

    \end{subfigure}%
    ~ 
    \begin{subfigure}[t]{0.20\textwidth}
        \centering
        \includegraphics[height=1.2in]{Files/Cages_Years_Matrix.jpg}

    \end{subfigure}

\end{figure}
\fi
 % Experiment 1

\section{Prediction of values about the data}

Some basic and general goals were defined before starting this phase, with the idea of "doing more if it's possible". Basically the main purpose was the one of, after the previous analysis, predict some values and evaluate the quality of the results.
This prediction system was not defined with some specific requirements, so the first main problem was to find a reliable, accurated and user-friendly way to predict and display prediction of values.

Since the current dataset can be considered like a time series, in this phase we will develop the data prediction system using an ARIMA machine implemented in python.

The ARIMA machine can be configured with several configurations, it allows you to have more accurated results; so the first thing was to find the right configuration of the ARIMA machine of each single input which we are interested to forecast.

General steps of this experiment:
\begin{enumerate}
\item Test 112 different configurations for each single input that we would like to forecast and report the results with each MAPE (Mean Average Percentage Error) values.
\item Run the ARIMA Machine with the best configuration for that particular input, that means the configuration which displayed the lowest MAPE during the previous testing part.
\item Collect real 2017 values of each single input.
\item Display forecasted 2017 values and real 2017 values.
\end{enumerate}

During the implementation of this system have been implemented other 3 subsystems:
- Evaluating system: used for evaluate different configurations of ARIMA machine
- Testing system: used for display the result on a dataset already known
- Future prediction system: used for predict values in the future
	
\newpage
\subsection{Evaluating System}
\textbf{Goal:}\\ Used for evaluate different configurations of ARIMA machine

\textbf{Requirements:}\\
There are not strict requirements needed. The input dataset could be as long as you want.

\textbf{Code implementation:}\\
The most important part of the code about the Evaluating System is the following.\\
Basically the method ARIMA() allows to train a model based on historic values (history) and a specific order (p,d,q). After that it's possible to call the method forecast() through the trained method and having some predictions like result.
\begin{lstlisting}
model = ARIMA(history, order=arima_order)
model_fit = model.fit(disp=0)
yhat = model_fit.forecast()[0]
\end{lstlisting}

This system will provide 112 different ARIMA configurations results for each single input, and in particular it will display the best ARIMA configuration, that is the one with the lower MAPE.

\textbf{Results:}\\
The system will display the MAPE between real value and predicted values for each single tested ARIMA machine, in particular the configuration that gives the best result.
All these results have been reported in a document and then also displayed with a 3D graphic that allows to see the MAPE value for each different order in input.



\begin{figure}[H]
	\raggedleft
	\makebox[\textwidth][c]{\includegraphics[width=0.7\textwidth]{Files/Cages_MAPE.jpg}}
    \caption{Graphic that displays different MAPE values for each ARIMA order.}
\end{figure}

 
 
\newpage
\subsection{Training System}
\textbf{Goal:}\\ Used for display the result on a dataset already known

\textbf{Requirements:}\\
Since this Traning System has been used mainly for train and test the current dataset, it need to have like input a dataset that follows the same format:
- Data content: 144 values, 1 value for each month from 2005 to 2016

\textbf{Code implementation:}\\
\begin{lstlisting}

\end{lstlisting}

\textbf{Results:}\\

\begin{figure}[H]
	\centering
    \makebox[\textwidth][c]{\includegraphics[width=1.5\textwidth]{Files/Cages_TRAIN.jpg}}
    \caption{Graphic that displays the predicted values from a particular ARIMA machine configuration and the historic real values.}
\end{figure}


\newpage
\subsection{Future Prediction System}
\textbf{Goal:}\\ Used for predict values in the future

\textbf{Requirements:}\\
- Input dataset:\\
- Data content:

\textbf{Code implementation:}\\
\begin{lstlisting}

\end{lstlisting}

\textbf{Results:}\\
\begin{figure}[H]
	\centering
    \makebox[\textwidth][c]{\includegraphics[width=1\textwidth]{Files/Cages_Predictions.jpg}}
    \caption{Graphic that display historic, future and predicted values of a input.}
\end{figure}

 % Experiment 2




