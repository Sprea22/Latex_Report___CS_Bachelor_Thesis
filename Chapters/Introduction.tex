%*******10********20********30********40********50********60********70********80

\chap{Introduction}

During the last few years we have witnessed an ever-increasing production of data in any kind of field and sectors all around the world.
For this reason an instrument or techniques for analyzing and understanding these data have been more and more needed, in order to extract useful information that might be used to improve business strategies or people's life condition.

Data Science is a recent launch field which contains processes and systems that could be used to extract knowledge from data, either structured or unstructured. Since the newness of this field, would be very interesting to test and evaluate different ways to apply daily technologies to its procedures and systems.

Python is a simple interpreted, object-oriented, high-level programming language that has a easy to learn syntax. It might provides a high productivity and severals modules and package.

The processes and systems provided from this field might be applied to areas of high economic interest, such as the Aquaculture industry in Norway. This business area is producing a big amount of data about every single locality or about national statistcs, but most of the time this data are difficult to understand and not analyzed at all.

\newpage

\section{Aim of the study}

The main purposes of this thesis are basically:
\begin{itemize} 
 \item Initial approach with Data Science field.
 
 \item Test and show Python potential in Data Science field.
 
 \item Report an initial analysis of data about Norwegian Salmon Farming.

 \end{itemize}
For achieve the goals reported above, this thesis will provide:
\begin{itemize} 
 \item 	Implementation and description of a procedure that can be used for realize a Python system able to do an automatic initial analysis of big datasets and display the obtained results.
 \item Implementation and description of a procedure that can be used for make a Python system able to predict future’s values using a regression model. 
 \end{itemize}


\newpage
\section{Initial Goals}

\textbf{1) Collect as much data about aquaculture in Norway as possible.}
\vspace{-5mm}
\begin{itemize}
 \setlength{\itemsep}{-5pt}
  \item Which kind of data is possible to obtain about aquaculture general statistics in Norway? Where is possible to find it? Are that available for everyone?
  \item Which kind of data is possible to obtain about aquaculture of single locations in Norway? Where is possible to find it? Are that available for everyone?
\end{itemize}

 
\textbf{2) Increase accessibility and availability of the data.}
\vspace{-5mm}
\begin{itemize}
 \setlength{\itemsep}{-5pt}
  \item How you can create a unique dataset that contains and summarize all the data previous collected?
  \item Which kind of structure allows to the total dataset to be more accessable and readadble than the original single sources?
\end{itemize}
 
\textbf{3) Analyze and display the data.}
\vspace{-5mm}
\begin{itemize}
 \setlength{\itemsep}{-5pt}
  \item Which kind of Python functions is possible to use for analyze and displaying data? 
  \item Which kind of requirements does it need and how is possible to implement it?
  \item Why Python could be a good solution for data analysis and displaying? 
  \item Which kind of relationships and patterns about the data is possible to identify using the result graphics? How is possible to identify it?
  \item How is possible to check out the data trend line? 
  \item Which kind of informations have been reported for future reuse? How it's possible to access it? (Informations such as correlation coefficients, trend line equations,..)
 \end{itemize}


\textbf{4) Extract information from the data.}
\vspace{-5mm}
\begin{itemize}
 \setlength{\itemsep}{-5pt}
  \item Which parameters about aquaculture in Norway are increasing? How fast are they increasing/decreasing? 
  \item How you can compare different parameters trend line?
  \item Which kind of correlations is possible to find out between different parameters? How is possible to show it? What is possible to extract from that?
 \end{itemize}
 
 \newpage
 
\textbf{5) Prediction of values about the data.}
\vspace{-5mm}
\begin{itemize}
 \setlength{\itemsep}{-5pt} 
  \item Which kind of Python utilities is possible to use for time series predictions?
  		\vspace{-3mm}
		\begin{itemize}
 		\setlength{\itemsep}{-5pt}	
		  \item How Python works for time series prediction systems implementation?
		  \item Which kind of accuracy it provides about the predicted values?
		  \item Would it be a good way for let the people get some experience with the machine learning field? 
		\end{itemize}
  \item Would be useful to have the possibility of forecasting some future data? 
  \item Which kind of data might be the most useful to know for people into the Aquaculture field?
	
 \end{itemize}

\textbf{6) Recommendations to future work and extra ideas.}
\vspace{-5mm}
\begin{itemize}
\setlength{\itemsep}{-5pt}
	\item How it could be possible to improve the Anaysis and Displaying system?
	\item How it could bee possible to improve the Forecasting system?
	\item Which kind of services is possible to provide using the collected informations and the implemented systems?
  		\vspace{-3mm}
		\begin{itemize}
 		\setlength{\itemsep}{-5pt}	
		  	\item How you can provide the analysis system like a service?
		  	\item How you can provide the prediction system like a service?
		\end{itemize}
 \end{itemize}

\section{Previous Works}


