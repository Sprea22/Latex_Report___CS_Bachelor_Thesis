%*******10********20********30********40********50********60********70********80

\chap{Introduction}

During the last few years we have witnessed an ever-increasing production of data in any sector all around the world.
For this reason instruments and techniques for analyzing and understanding these data are becoming more and more indispensable, in order to extract useful information that might be used to improve business strategies or people's life condition.

Data Science is a recent launch field which contains processes and systems that could be used to extract knowledge from data, either structured or unstructured. Since the newness of this field, would be very interesting to test and evaluate different ways to apply daily technologies to its procedures and systems.

Python is a simple interpreted, object-oriente and high-level programming language that has a easy to learn syntax.
Since it provides severals modules and package, the use of Python during a Data Science process could be very productive.

The processes and systems which belong to the Data Science field might be applied to high-interest  economic areas, such as the Aquaculture industry in Norway.
This business, in particular the Norwegian salmon farming, has a big economic repercussions on the country, and at the same time is producing a huge amount of data, so it would be very helpful to restructure and analyze it.

This thesis will contributes providing a documented implementation of an analysis, displaying and prediction system using Python applied to the Norwegian salmon farming.

\newpage

\section{Aim of the study}
\vspace{-5mm}
The focus of this study will be on:
\begin{itemize} 
 \item Initial approach with Data Science field, in order to investigate and document possible techniques, methods and approaches.
 
 \item Testing Python potential in Data Science field, describing implementation procedures and reporting pro and cons. 
 
 \item Report the initial analysis and displaying results about Norwegian salmon farming, in order to provide structured, described and readable data that might be used for future works.
\end{itemize}

\section{Research Objectives}
\vspace{-5mm}
The above aim will be accomplished by fulfilling the following research objectives:

\textbf{1) Collect as much data about aquaculture in Norway as possible.}
\vspace{-5mm}
\begin{itemize}
 \setlength{\itemsep}{-5pt}
  \item Which kind of data is possible to obtain about aquaculture general statistics in Norway? Where is possible to find it? Are that available for everyone?
  \item Which kind of data is possible to obtain about aquaculture of single locations in Norway? Where is possible to find it? Are that available for everyone?
\end{itemize}

 
\textbf{2) Increase accessibility and availability of the data.}
\vspace{-5mm}
\begin{itemize}
 \setlength{\itemsep}{-5pt}
  \item How you can create a unique dataset that contains and summarize all the data previous collected?
  \item Which kind of structure allows to the total dataset to be more accessable and readadble than the original single sources?
\end{itemize}
 
\textbf{3) Analyze and display the data.}
\vspace{-5mm}
\begin{itemize}
 \setlength{\itemsep}{-5pt}
  \item Which kind of Python functions is possible to use for analyze and displaying data? 
  \item Which kind of requirements does it need and how is possible to implement it?
  \item Why Python could be a good solution for data analysis and displaying? 
  \item Which kind of relationships and patterns about the data is possible to identify using the result graphics? How is possible to identify it?
  \item How is possible to check out the data trend line? 
  \item Which kind of informations have been reported for future reuse? How it's possible to access it? (Informations such as correlation coefficients, trend line equations,..)
 \end{itemize}

\newpage

\textbf{4) Extract information from the data.}
\vspace{-5mm}
\begin{itemize}
 \setlength{\itemsep}{-5pt}
  \item Which parameters about aquaculture in Norway are increasing? How fast are they increasing/decreasing? 
  \item How you can compare different parameters trend line?
  \item Which kind of correlations is possible to find out between different parameters? How is possible to show it? What is possible to extract from that?
 \end{itemize}
 
\textbf{5) Prediction of values about the data.}
\vspace{-5mm}
\begin{itemize}
 \setlength{\itemsep}{-5pt} 
  \item Which kind of Python utilities is possible to use for time series predictions?
  		\vspace{-3mm}
		\begin{itemize}
 		\setlength{\itemsep}{-5pt}	
		  \item How Python works for time series prediction systems implementation?
		  \item Which kind of accuracy it provides about the predicted values?
		  \item Would it be a good way for let the people get some experience with the machine learning field? 
		\end{itemize}
  \item Would be useful to have the possibility of forecasting some future data? 
  \item Which kind of data might be the most useful to know for people into the Aquaculture field?
	
 \end{itemize}

\textbf{6) Recommendations to future work and extra ideas.}
\vspace{-5mm}
\begin{itemize}
\setlength{\itemsep}{-5pt}
	\item How it could be possible to improve the Anaysis and Displaying system?
	\item How it could bee possible to improve the Forecasting system?
	\item Which kind of services is possible to provide using the collected informations and the implemented systems?
  		\vspace{-3mm}
		\begin{itemize}
 		\setlength{\itemsep}{-5pt}	
		  	\item How you can provide the analysis system like a service?
		  	\item How you can provide the prediction system like a service?
		\end{itemize}
 \end{itemize}


\section{Previous Works}
\vspace{-5mm}
The implementation of the prediction system in this thesis was based on a previous work, which provides a basic implementation of a forecasting system with Python.\\
That particular work was showing how to create a general ARIMA Model for Time Series Forecasting with Python. During this study that implementation has been improved, customized and applied to the current context.

The previous work source website is named "machinelearningmastery.com", and here's the reference: \cite{previousWork}


