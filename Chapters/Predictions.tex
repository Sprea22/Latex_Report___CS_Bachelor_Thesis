\chapter{Prediction System}
The main goal during this phase was to find a way to implement a forecast system in Python.
Since the datasets used during this thesis could be considered as a time series, was decide to implement and test an Autoregressive Integrated Moving Average (ARIMA) model. \\
Since there are several possible configurations for fit an ARIMA model, is important to find the right one to use with each input dataset because it would allows to have much better prediction results.
In order to find the best ARIMA configuration there are different methods and procedure, like one of the most known that is "Box–Jenkins method"\footnote{Check out the Box–Jenkins method at the current link: \\ \url{https://en.wikipedia.org/wiki/Box\%E2\%80\%93Jenkins_method}}. In this study was decied to use an easier method, in order to have a first approach with this system and a general idea about the problem. It basically consists in testing different ARIMA model configurations for the same input dataset and then check the results.\\
For this reason during this phase of the work have been implemented 2 different subsytems for two different purposes:
\begin{enumerate}
\item Evaluating System
\item Prediction System
\end{enumerate}

\newpage
\section{Evaluating System}
\textbf{Goal:}\\ 
Used for evaluate different configurations of ARIMA machine. \\ It tests 112 different configurations for the current input and report in a document the tested configurations with the corresponding MAPE (Mean Average Percentage Error) value.

\textbf{Requirements:}\\
There are not any kind of needed requirements. It's possible to use this system on dataset of arbitrary length.

\textbf{Code implementation:}\\
The most important part of the code about the Evaluating System is the following.\\
Basically the method ARIMA() allows to train a model based on historic values (history) and a specific order (p,d,q). After that it's possible to call the method forecast() through the trained model and having some predictions like result.
\begin{lstlisting}
model = ARIMA(history, order=arima_order)
model_fit = model.fit(disp=0)
yhat = model_fit.forecast()[0]
\end{lstlisting}

More in the specific, the 112 different ARIMA configurations that were tested are all the possible combinations between the following three parameters values:
\begin{lstlisting}
p_values = [0, 1, 2, 4, 6, 8, 10]
d_values = [0, 1, 2, 3]
q_values = [0, 1, 2, 3]
\end{lstlisting}

It's possible to check out the full implemented code in the appendice: [\ref{Evaluating_System}]

\newpage

\textbf{Results:}\\
The system will report in a document the MAPE between real value and predicted values for each of the 112 tested ARIMA machine.
During this part of the work was also create a script for execute an ARIMA evaluation about each single paramater of each single dataset, but once executed it required too much time for complete all the evaluations. \\
For that reason have been reported in a document just some relevant evaluations result. The table below here gaves an example about how the documents look like and also report some results about specific paramater that are going to be used in the next phase.

 \begin{table}[ht]
\makebox[\textwidth][c]{
\resizebox{1\textwidth}{!}{
    \begin{tabular}{ | l | l | l | l |}
            \hline
\textbf{Input} 	& \textbf{Parameter}	& \textbf{ARIMA Conf} 	& \textbf{MAPE result} 	\\ \hline
Finnmark & feedConsumption				& (6, 1, 0) 			& 13.771\% 		 		\\ \hline	
Hordaland & feedConsumption				& (8, 0, 0)  			& 6.811\%				\\ \hline				
Troms & feedConsumption					& (2, 0, 0)  			& 11.593\% 				\\ \hline
Nordland & feedConsumption				& (6, 0, 0)  			& 12.741\%		 	\\ \hline
Norway0714 & feedConsumption			& (6, 1, 0)  			& 7.296\%  				\\ \hline
    \end{tabular}}}
         \caption{MAPE Results for some particular dataset about the parameter "feedConsumption"}   
   \label{table: MAPE_Results_feedConsumption} 
\end{table}     

\newpage

\makebox[\textwidth][c]{
\resizebox{1.3\textwidth}{!}{
\begin{tabular}{|c|c|c|c|c|c|c|c|c|c|c|c|}
\hline
\multirow{3}{*}{Predicted Months} & \multicolumn{3}{c|}{Finnmark : (6,1,0)} & \multicolumn{3}{c|}{Hordaland : (8, 0, 0)} & \multicolumn{3}{c|}{Troms : ( , , )}\\
\cline{2-10}
 & Real & Pred & Error & Real & Pred & Error & Real & Pred & Error\\
\hline
 January 2015 & 7571.038 	& 	& \% 		& 15039.948 & 	 & \% &  12534.708		& 	 & \% \\
\hline
 February 2015 & 4947.371 	& 	& \% 		& 12432.481 & 	 & \% &  9098.07		& 	 & \%  \\
 \hline
 March 2015 & 4686.465 		& 	& \% 		& 12523.293 & 	 & \% &  9381.861		& 	 & \% \\
 \hline
 April 2015 & 4263.749		 & 	 & \%		& 14231.265 & 	 & \% & 9493.93			& 	& \%  \\
 \hline
 May 2015 & 5472.568 		& 	& \%		& 14543.612 & 	 & \% &	11215.47		& 	 & \%  \\
 \hline
 June 2015 & 8270.913 		& 	& \%		& 15314.691 & 	 & \% & 16370.732		& 	 & \%\\
 \hline
 July 2015 & 12356.162 		& 	& \%		& 19617.51 & 	 & \% & 21041.845		& 	 & \%  \\
 \hline
 August 2015 & 15877.981 	& 	 & \%		& 26429.183 & 	 & \% & 26981.845		& 	 & \% \\
 \hline
 September 2015 & 15382.371 & 	& \%		& 28152.314 & 	 & \% & 25229.126		&  	& \% \\
 \hline
 October 2015 & 15039.109 	&	& \%		& 26869.594 & 	 & \% & 27327.632		&  &  \% \\
 \hline
 November 2015 & 11453.453 	& 	 & \%		& 23914.498 &	 & \% & 22476.374		&  	& \% \\
 \hline
 December 2015 & 9450.334 	& 	 & \%		& 20332.347 & 	 & \% & 17239.549		& 	 &  \% \\
 \hline
% etc. ...
\end{tabular}  }}

\makebox[\textwidth][c]{
\resizebox{1\textwidth}{!}{
\begin{tabular}{|c|c|c|c|c|c|c|c|c|}
\hline
\multirow{3}{*}{Predicted Months} & \multicolumn{3}{c|}{Nordland : (,,)} & \multicolumn{3}{c|}{Norway : (, , )} \\
\cline{2-7}
 & Real & Pred & Error & Real & Pred & Error \\
\hline
 January 2015 & 7571.038 	& 	& \% 		& 15039.948 & 	 & \%  \\
\hline
 February 2015 & 4947.371 	& 	& \% 		& 12432.481 & 	 & \%   \\
 \hline
 March 2015 & 4686.465 		& 	& \% 		& 12523.293 & 	 & \%   \\
 \hline
 April 2015 & 4263.749		 & 	 & \%		& 14231.265 & 	 & \%   \\
 \hline
 May 2015 & 5472.568 		& 	& \%		& 14543.612 & 	 & \% 	\\
 \hline
 June 2015 & 8270.913 		& 	& \%		& 15314.691 & 	 & \%  \\
 \hline
 July 2015 & 12356.162 		& 	& \%		& 19617.51 & 	 & \%   \\
 \hline
 August 2015 & 15877.981 	& 	 & \%		& 26429.183 & 	 & \% \\
 \hline
 September 2015 & 15382.371 & 	& \%		& 28152.314 & 	 & \%  \\
 \hline
 October 2015 & 15039.109 	&	& \%		& 26869.594 & 	 & \%  \\
 \hline
 November 2015 & 11453.453 	& 	 & \%		& 23914.498 &	 & \%  \\
 \hline
 December 2015 & 9450.334 	& 	 & \%		& 20332.347 & 	 & \%  \\
 \hline
% etc. ...
\end{tabular}  }}
 
\newpage
\section{Prediction System}
\textbf{Goal:}\\ 
This system has two main goals:
\vspace{-5mm}
\begin{itemize}
 \setlength{\itemsep}{-5pt} 
\item Predict some future value with the a specific ARIMA configuration.
\item Display the historic data together with real future values and predicted future values.
\end{itemize}

\textbf{Requirements:}\\
The only requirement is to use in a correct way this system is that the real future values are available, in order to compare it with the predicted one. It's possible to use this system on input dataset of arbitrary length.


\textbf{Code implementation:}\\
The method ARIMA() allows to train a model based on historic values (history) and a specific order (p,d,q). After that it's possible to call the method forecast() through the trained model with a "int" parameter that represents the desired number of predictions that have to be calculated.

\begin{lstlisting}
model = ARIMA(dataset, order=order)
model_fit = model.fit(disp=0)
forecast = model_fit.forecast(int(sys.argv[3]))[0]
\end{lstlisting}

Once calculated and saved the predictions in a document, the system will basically display on the same graphic "realVaues" that contains the real future values, "predFuture" that contains the predicted future values and "series" that contains the dataset historic values.
\begin{lstlisting}
ax.plot(realValues, "g", label='Real 2015 Values', linewidth=2)
ax.plot(predFuture, "r", label='Prediction 2015 values', linewidth=2)
ax.plot(series, "b", label='Historic values', linewidth=2)
\end{lstlisting}

It's possible to check out the full implemented code in the appendice: [\ref{Prediction_System}]

\newpage

\textbf{Results:}\\
This system will automatically generate a document that contain:
\vspace{-5mm}
\begin{itemize}
 \setlength{\itemsep}{-5pt} 
\item Real future values
\item Predicted future values
\item MAPE between each prediction and the corresponding real value.
\end{itemize}

And then it provides also the possibility to visualize the historic, real and predicted future values on the same graphic, that looks like the following example:

\begin{figure}[H]
    \makebox[\textwidth][c]{\includegraphics[width=1.3\textwidth]{Files/Finnmark-feedConsumption_pred.pdf}}
    \caption{Graphic that display historic, real and predicted future values of a input.}
\end{figure}


\newpage



